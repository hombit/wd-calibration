\begin{table*}
\scriptsize
\begin{centering}
%\Rotatebox{90}{%
\begin{tabular}{l|cc|cccccc}
\hline
\hline
Object & R.A. & $\delta$  & \textit{F275W} & \textit{F336W} & \textit{F475W} & \textit{F625W} & \textit{F775W} & \textit{F160W} \\
\hline
G191-B2B & 05:05:30.61 & +52:49:51.96 & 10.4904 (1) & 10.8902 (1) & 11.4988 (1) & 12.0307 (1) & 12.4514 (1) & 13.8853 (2) \\
GD153 & 12:57:02.34 & +22:01:52.68 & 12.2016 (2) & 12.5679 (1) & 13.0998 (2) & 13.5976 (1) & 14.0017 (1) & 15.4139 (2) \\
GD71 & 05:52:27.61 & +15:53:13.75 & 11.9888 (1) & 12.3360 (1) & 12.7988 (1) & 13.2790 (1) & 13.6720 (1) & 15.0676 (2) \\
\hline
SDSS-J010322 & 01:03:22.19 & -00:20:47.73 & 18.1952 (4) & 18.5268 (5) & 19.0833 (5) & 19.5686 (5) & 19.9648 (6) & 21.3552 (12) \\
SDSS-J022817 & 02:28:17.17 & -08:27:16.41 & 19.5183 (8) & 19.7152 (10) & 19.8151 (7) & 20.1690 (7) & 20.5014 (6) & 21.7371 (17) \\
SDSS-J024854 & 02:48:54.96 & +33:45:48.30 & 17.8285 (4) & 18.0400 (6) & 18.3696 (3) & 18.7459 (3) & 19.0773 (2) & 20.3400 (6) \\
SDSS-J072752 & 07:27:52.76 & +32:14:16.10 & 17.1636 (3) & 17.4715 (3) & 17.9933 (3) & 18.4567 (2) & 18.8370 (3) & 20.2166 (7) \\
SDSS-J081508 & 08:15:08.78 & +07:31:45.80 & 18.9505 (6) & 19.2635 (8) & 19.7162 (5) & 20.1838 (5) & 20.5794 (6) & 21.9616 (24) \\
SDSS-J102430 & 10:24:30.93 & -00:32:07.03 & 18.2606 (18) & 18.5143 (4) & 18.9042 (5) & 19.3174 (4) & 19.6649 (10) & 20.9905 (13) \\
SDSS-J111059 & 11:10:59.43 & -17:09:54.10 & 17.0406 (3) & 17.3544 (4) & 17.8668 (3) & 18.3135 (2) & 18.6887 (2) & 20.0566 (5) \\
SDSS-J111127 & 11:11:27.30 & +39:56:28.00 & 17.4429 (4) & 17.8298 (6) & 18.4206 (3) & 18.9390 (4) & 19.3441 (3) & 20.7975 (9) \\
SDSS-J120650 & 12:06:50.41 & +02:01:42.46 & 18.2397 (4) & 18.4888 (4) & 18.6719 (4) & 19.0601 (3) & 19.4112 (7) & 20.7027 (9) \\
SDSS-J121405 & 12:14:05.11 & +45:38:18.50 & 16.9401 (2) & 17.2827 (2) & 17.7606 (2) & 18.2362 (3) & 18.6292 (2) & 20.0378 (4) \\
SDSS-J130234 & 13:02:34.44 & +10:12:39.01 & 16.1879 (2) & 16.5216 (2) & 17.0364 (2) & 17.5140 (2) & 17.9037 (2) & 19.3031 (4) \\
SDSS-J131445 & 13:14:45.05 & -03:14:15.64 & 18.2577 (4) & 18.5969 (5) & 19.1018 (5) & 19.5668 (5) & 19.9553 (9) & 21.3284 (12) \\
SDSS-J151421 & 15:14:21.27 & +00:47:52.79 & 15.1100 (2) & 15.3907 (2) & 15.7090 (2) & 16.1202 (2) & 16.4712 (1) & 17.7870 (4) \\
SDSS-J155745 & 15:57:45.40 & +55:46:09.70 & 16.4999 (2) & 16.8766 (2) & 17.4702 (3) & 17.9917 (2) & 18.3880 (2) & 19.8343 (5) \\
SDSS-J163800 & 16:38:00.36 & +00:47:17.81 & 18.0158 (8) & 18.3177 (4) & 18.8399 (5) & 19.2808 (3) & 19.6605 (5) & 20.9963 (9) \\
SDSS-J181424 & 18:14:24.13 & +78:54:02.90 & 15.7913 (2) & 16.1213 (2) & 16.5441 (2) & 17.0056 (2) & 17.3926 (1) & 18.7857 (2) \\
SDSS-J210150 & 21:01:50.66 & -05:45:50.97 & 18.0681 (4) & 18.3344 (4) & 18.6560 (3) & 19.0636 (2) & 19.4140 (4) & 20.7396 (8) \\
SDSS-J232941 & 23:29:41.33 & +00:11:07.80 & 17.9434 (4) & 18.1090 (4) & 18.1607 (6) & 18.4697 (3) & 18.7753 (7) & 19.9949 (6) \\
SDSS-J235144 & 23:51:44.29 & +37:55:42.60 & 17.4494 (4) & 17.6619 (3) & 18.0751 (3) & 18.4595 (3) & 18.7868 (2) & 20.0747 (4) \\
\tableline
\tableline
\hline
\end{tabular}
\footnotesize{\tablecomments{Coordinates are reported with epoch J2000. Apparent magnitudes measured through each passband are reported in columns with names corresponding to the passband names, followed by the uncertainties parenthetically. All measurements are rounded to a tenth of a millimag. Magnitudes are tabulated on the AB system, and are followed by the parameters of our model of the photometric observations. The zeropoint in each passband Z are determined from the difference between the synthetic magnitudes of the three CALSPEC primary standards and their measured instrumental magnitudes as described in \S\ref{sec:photmodel}. The offsets between cycle 20 and 22 \Delta Z_{\text{C20}} are determined from all stars with observations in both cycles. All observations are used to infer the dispersion \sigma_{\text{int}} and the degrees of freedom \nu of the Student-t distribution, which describe the photometric repeatability and the outliers caused by cosmic rays respectively in each passband. Higher values of \nu indicate increasing Gaussianity. The parameter \nu is dimensionless, and its value and error are reported to three decimal places. }
\tablenotetext{*}{\citetalias{Narayan16} showed that SDSS-J203722 exhibited time-variable emission in the cores of the Balmer lines and excluded this object from their analysis. Additionally, we exclude SDSS-J041053, WD0554, and SDSS-J172135 in this work - see \S\ref{sec:excluded} for details. Their measured apparent magnitudes are listed here for completeness.}}
\caption{Apparent AB Magnitudes and Photometric Uncertainties of the Network of DA White Dwarfs and CALSPEC Primary Standards\label{table:phot}}
\end{centering}
\end{table*}
